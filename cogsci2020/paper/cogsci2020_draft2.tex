% 
% Annual Cognitive Science Conference
% Sample LaTeX Paper -- Proceedings Format
% 

% Original : Ashwin Ram (ashwin@cc.gatech.edu)       04/01/1994
% Modified : Johanna Moore (jmoore@cs.pitt.edu)      03/17/1995
% Modified : David Noelle (noelle@ucsd.edu)          03/15/1996
% Modified : Pat Langley (langley@cs.stanford.edu)   01/26/1997
% Latex2e corrections by Ramin Charles Nakisa        01/28/1997 
% Modified : Tina Eliassi-Rad (eliassi@cs.wisc.edu)  01/31/1998
% Modified : Trisha Yannuzzi (trisha@ircs.upenn.edu) 12/28/1999 (in process)
% Modified : Mary Ellen Foster (M.E.Foster@ed.ac.uk) 12/11/2000
% Modified : Ken Forbus                              01/23/2004
% Modified : Eli M. Silk (esilk@pitt.edu)            05/24/2005
% Modified : Niels Taatgen (taatgen@cmu.edu)         10/24/2006
% Modified : David Noelle (dnoelle@ucmerced.edu)     11/19/2014
% Modified : Roger Levy (rplevy@mit.edu)     12/31/2018



%% Change "letterpaper" in the following line to "a4paper" if you must.

\documentclass[10pt,letterpaper]{article}

\usepackage{cogsci}
\usepackage{url}

%\cogscifinalcopy % Uncomment this line for the final submission 


\usepackage{pslatex}
\usepackage{apacite}
\usepackage{float} % Roger Levy added this and changed figure/table
                   % placement to [H] for conformity to Word template,
                   % though floating tables and figures to top is
                   % still generally recommended!

%\usepackage[none]{hyphenat} % Sometimes it can be useful to turn off
%hyphenation for purposes such as spell checking of the resulting
%PDF.  Uncomment this block to turn off hyphenation.
\usepackage{graphicx}


%\setlength\titlebox{4.5cm}
% You can expand the titlebox if you need extra space
% to show all the authors. Please do not make the titlebox
% smaller than 4.5cm (the original size).
%%If you do, we reserve the right to require you to change it back in
%%the camera-ready version, which could interfere with the timely
%%appearance of your paper in the Proceedings.

\title{Birds and Words: Exploring environmental influences on folk categorization}
 
\author{{\large \bf Joshua T. Abbott (joshua.abbott@unimelb.edu.au)} \\
 {\large \bf Charles Kemp (ckemp@unimelb.edu.au)} \\
  School of Psychological Sciences,  \\
  University of Melbourne, 3010, Australia}



\begin{document}

\maketitle


\begin{abstract}
Questions of how categories of natural kinds are named have a long history in anthropology and psychology. There has been a particular debate over the influence of the external environment on how people partition the world. Recent large-scale digital datasets allow us to test old questions in a new light. In this paper we use frequency of occurrence and physical size of birds to examine the naming patterns of a folk categorization system. First, we explore the extensions of categories, and investigate whether these environmental factors predict whether a bird is named and how they are grouped with other birds by the same label. We then explore how the forms of category labels are predicted by the environmental factors. We demonstrate that the recent development and release of multiple digital datasets allows us to investigate old questions with a new perspective.

\textbf{Keywords:} 
ethnobiology; categorization; bird naming
\end{abstract}


\section{Introduction}

Languages around the world include rich systems of names for plants and animals, and each system can be viewed as the outcome of a natural experiment in which generations of speakers have organized their local environment into categories. A classic line of work in cognitive anthropology addresses the question of how named categories reflect the structure of the local environment~\cite{berlin2014ethnobiological}. One prominent theme is that folk taxonomies often align well with Western scientific taxonomies, suggesting that folk taxonomies are shaped more by environmental structure than by the idiosyncratic needs and concerns of a particular culture. 

Much of the cognitively-oriented work on folk biology took place last century, and in recent years new data sets have made it possible to characterize the structure of the environment in ways that were previously difficult or impossible~\cite{sullivan2009ebird,wilman2014eltontraits}. Here we draw on these resources to revisit the classic question of the relationship between named categories and the environment. We focus on birds in particular, and begin by compiling properties of the bird species in a given area (e.g.\ how big is each species, and how often is it observed?) We then study how these properties relate to named bird categories in the local language. In particular, we ask whether the frequency of a species influences whether the species is named, and if so whether frequency influences the form of the name for that species and how many other species it is grouped with. 

The effects of frequency on categorization have been previously studied in the psychological literature~\cite{parducci83,nosofsky88,barsalouhl98}.  One relevant finding is that categories tend to be relatively broad in low-frequency regions of stimulus space, but relatively narrow in regions including frequently encountered stimuli~\cite{parducci83}. We might therefore predict that bird species encountered frequently are more likely to be assigned to their own distinctive categories.  

Our focus on frequency also connects with a prominent debate between \emph{intellectualist} ~\cite{berlin2014ethnobiological} and \emph{utilitarian} ~\cite{hunn1982utilitarian} accounts of folk classification. The intellectualist view holds that named categories reflect ``fundamental biological discontinuities'' that are perceptually salient (Berlin, 1992 p 53)~\nocite{berlin2014ethnobiological}, and assigns a minimal role to freqeuncy. The utilitarian view emphasizes ways in which categories are useful for a given culture, and naturally accommodates frequency effects because assigning a label to a category is especially worthwhile if there are many occasions to use it.

The next section introduces the data sets that we use, and we then address two broad questions. First, we focus on category extensions, and ask whether environmental factors predict whether a species is named, and how the set of named species is organized into groups. Second, we focus on category labels, and ask whether environmental factors predict the relative lengths of category labels, and which labels have the structure of unmarked prototypes. 


\section{Data sets}

The literature contains detailed folk classifications of birds from several languages around the world, and we focus here on named bird categories from Zapotec \cite{hunn2008zapotec}, a language spoken in Oaxaca, Mexico.  We used two data sets that characterize the frequency and size of bird species found in Oaxaca, and a third that specifies how these species are organized into named categories.

\subsection{Frequency data}

Our frequency data are drawn from eBird, a citizen-science based bird observation network managed by the Cornell Lab of Ornithology \cite{sullivan2009ebird}.  eBird data are contributed by bird lovers (both professional and amateur) who use the site to record the time and place of bird sightings.  We used data from just the region containing the state of Oaxaca, Mexico\footnote{We used all eBird observation of frequency from the Basic Dataset (EBD) on https://ebird.org/data/download, last accessed January 24, 2020.}. An observer who sees a group of 5 vultures may record both the species (e.g.\ \emph{Cathartes aura}) and the number of birds in the group (5), but we treated each case like this as a single observation of the species in question. Our data for Oaxaca include 660,223 unique observations of 922 distinct species. 

We will take eBird counts as a very rough proxy for the frequency with which a species is encountered in the course of regular life. The fact that nocturnal species will tend to have lower counts than equally common diurnal species is therefore a strength of the data rather than a limitation. eBird, however,  does not provide an unbiased measure of frequency in everyday life. As a group, eBird contributors are more interested in some species than others, and counts for rare but iconic species (e.g.\ the Bald eagle in the USA) will overestimate the frequency with which they are encountered relative to other species. Even so, eBird is a valuable resource that allows rough estimates of a variable (frequency) that would otherwise be extremely difficult to measure.  


\subsection{Size data}
% Bird weights as an aid in taxonomy \cite{amadon1943bird}.
Beyond frequency it is plausible that physical and behavioral characteristics of birds both influence folk categorization. \citeA{hunn1999size} has documented that smaller species are more likely to be lumped together into large categories, and larger species are more likely to be given distinct names. Following his lead we evaluate bird size as an influence on categorization, and use size data from EltonTraits~\cite{wilman2014eltontraits} which includes information on key attributes for all 9993 extant bird species, including those from Oaxaca.  We use the body mass variable, separately sourced from \cite{dunning2007crc}, which is defined as the geometric mean of average values provided for both sexes. Beyond body mass EltonTraits includes variables related to diet types, foraging strata, and activity patterns, and future studies can explore whether and how these variables influence naming. 

%\subsection{Phenotypic neighborhood}
%% Bird weights as an aid in taxonomy \cite{amadon1943bird}.
%
%An obvious factor affecting folk classification is that species are more likely to be grouped together if they look similar. We do not have detailed information about the appearance of species in our data set, but work with a similarity space (Figure~\ref{fig-morphospace}) that roughly captures similarity in physical appearance. \cite{pigot} generated a 9-dimensional morphospace based on body mass in addition to 8 variables related to beak shape and body shape, and we mapped their data down to the 
% 2 dimensional space in Figure~\ref{fig-morphospace} by running tSNE \cite{XXX}, an algorithm for dimensionality reduction. As with all tSNE solutions, the axes (and axis units) in Figure~\ref{fig-morphospace} cannot be precisely interpreted, but the x axis corresponds fairly closely to body mass (correlation is 0.96), and the isolated group of points at the far left ( \textit{dz\v{i}n\b{g}}) includes hummingbirds.  We used the space in Figure~\ref{fig-morphospace} to derive a neighborhood size score that counts the number of species within a distance of 0.5 units. The mean neighborhood size across all species in Figure~\ref{fig-morphospace} is 1.5, but the mean for the hummingbirds alone is 3.7.
%
%\begin{figure}[hbt!]
%  \begin{center}
%    \includegraphics[width=3.375in]{./figures/morphospace.png}
%        \caption{Similarity space generated by applying tSNE to morphological data (e.g.\ beak shape, body shape and body mass). Colored points show species belonging to 7 Zapotec categories, and the remaining named species are shown as open grey circles.}
%        \label{fig-morphospace}
%  \end{center}
%\end{figure}


\subsection{Naming data}

Our naming data are based on a detailed folk taxonomy provided by \citeA{hunn2008zapotec} of the Zapotec language, from his fieldwork in San Juan Gb\"{e}\"{e}, a small village in Oaxaca, Mexico\footnote{ Also available online at \url{http://faculty.washington.edu/hunn/zapotec/z5.html}}. The data set includes a scientific name,  a folk-specific name and a folk-generic name for each species listed. For example, \emph{ Colibri thalassinus} (Mexican violetear) is named \textit{dz\v{\i}n\b{g}-y\v{a}-gu\`{i}} (mountain hummingbird) at the folk-specific level and \textit{dz\v{\i}n\b{g}} (hummingbird) at the folk-generic level.  According to Hunn's taxonomy the folk-generic category \textit{dz\v{\i}n\b{g}} (hummingbird) includes 14 different species. These 14 species are partitioned into 4 categories at the folk-specific level: \textit{dz\v{\i}n\b{g}}, \textit{dz\v{\i}n\b{g}-d\'{a}n-y\v{a}-gu\`{i}}, \textit{dz\v{\i}n\b{g}-gu\'{e}}, and \textit{dz\v{\i}n\b{g}-y\v{a}-gu\`{i}}.
As this example suggests, in some cases the folk-specific and folk-generic names for a species are identical: for example, 
 \emph{ Amazilia Beryllina} is known simply as \textit{dz\v{\i}n\b{g}} (hummingbird) at the folk-specific level.  Following Hunn's usage any folk-specific category (e.g.\ the one that includes \emph{ Amazilia Beryllina}) with the same label as the folk-generic category to which it belongs will be called an \emph{unmarked prototype}.
% JTA: I'm uncertain about this last sentence -- I think we want to be clearer about what an unmarked prototype is.

In total Hunn's taxonomy includes 152 species that are organized into 94 distinct folk-specific categories, which in turn are organized into 68 folk-generic categories.  The scientific species labels given by Hunn  did not always match those used by our other sources of data (eBird and EltonTraits).  We took the Clements checklist (used by eBird) as our gold standard \cite{clements2007clements}, and some manual preprocessing was required to align the labels used by all three resources.\footnote{All data and analysis code will be available on author's GitHub site.}

%We then linked the three resources to create a single data set with nam This resulted in 152 bird species named in Zapotec with corresponding frequencies and masses


\section{Analysis of category extensions}

Given the three data sets just described we ask whether the frequency and body mass data can predict aspects of Hunn's naming data.  We focus first on the extensions of folk categories, and subsequently consider the labels or names given to these categories. 

Our first analysis considers whether frequency and mass can predict whether a species is likely to be named. Intuitively, one might expect that common species are more likely to be named, and that larger species are especially salient perceptually and therefore more likely to be named. Most descriptions of folk classification systems in the literature do not systematically describe any species found in the local area that are \emph{not} named by the local people. Our eBird data, however, include species that were documented in Oaxaca but not included in Hunn's taxonomy. Some of these species are probably rarely if ever seen in the village (San Juan Gb\"{e}\"{e}) where Hunn carried out his fieldwork. We expect, however, that some species missing from the taxonomy would be occasionally encountered in San Juan Gb\"{e}\"{e}. 

\begin{figure*}[hbt!]
  \begin{center}
    \includegraphics[width=.95\textwidth]{./figures/birdfreqmass-violinplots.png}
        \caption{Frequency densities (left) and Mass densitities (right) for species named in Zapotec, species found in Oaxaca (OAX) but missing from the Zapotec taxonomy, and all species in OAX.}
        \label{fig-birdfreqmassviolin}
  \end{center}
\end{figure*}

Figure~\ref{fig-birdfreqmassviolin} shows distributions of frequency and size for birds with and without Zapotec names.  As expected, on average birds that are named tend to be more frequent than birds that are not named. The mass distributions for named and unnamed birds, however, are very similar. To confirm these impressions we ran a logistic regression including log frequency and log mass as predictors of a binary variable that indicates whether a species was named. The estimated coefficients were 0.84 (log frequency) and -0.03 (log mass). Based on these coefficients, if the probability that a species would receive a name were initially 0.5, increasing the log frequency of the species by one unit would increase the naming probability to 0.65, and increasing the log mass by one unit would decrease the naming probability to 0.49. We compared the full logistic regression model to alternatives that removed either log frequency or log mass as a predictor, and found that removing log frequency produced a significant impairment ($p < 0.01$), but removing log mass did not ($p = 0.59$).  Akaike information criterion (AIC) scores supported the conclusion that the model with log frequency but without log mass is the best among the three. 

% mention that information about unnamed species not systematically included in folk taxonomies




% 
% NEEDS WORK STARTING HERE
% 
% 
\subsection{Category organization}


Next we explore how named species are organized into categories. \citeA{hunn1999size} proposed that the size of an organism is influential in folk categorization and developed a measure which examines the degree to which the organisms are recognized taxonomically in the folk system based on size. 

We investigate a variant of Hunn's analyses, simply exploring how many folk generic category members each named species in Zapotec shares. For example, the folk-generic name for hummingbirds is \textit{dz\v{i}n\b{g}}, which covers 14 species of bird in Zapotec. We analyze this using both mass and frequency as predictors of how many species are grouped by name. 

\begin{figure*}[!ht]
  \begin{center}
    \includegraphics[width=0.95\textwidth]{./figures/ssrr-singlespecies.png}
        \caption{Category organization plots for both frequency (left column) and mass (right column). Category organization is measured as the number of other species a bird is grouped with under the same folk-generic name. Each point represents a bird species named in Zapotec.}
        \label{fig-ssrr}
  \end{center}
\end{figure*}

Figure~\ref{fig-ssrr} presents plots for both frequency (left column) and mass (right column). Here we find that mass rather than frequency is a better predictor of category grouping. The $R^2$ for frequency is 0.01, and for mass is 0.20. The coefficients of a linear regression suggest that mass is a stronger predictor than frequency. Dropping either predictor, however, leads to a model fit that is significantly worse ($p < 0.01$) for mass and ($p = 0.03$) for frequency. AIC values support the same conclusion.

Overall, these findings support Hunn's conclusion that size can predict the grouping behavior in folk categorization, while providing only weak evidence that frequency can do the same.


\section{Analysis of category labels}

We have explored how frequency and mass effect category extensions: whether or not a species is named, and how they are grouped with other species under the same name. We now investigate whether these environmental variables influence category labels, that is, the form of the name for that species. We explore three simple lexical properties of the Zapotec bird labels: the word length, whether the word is a compound or monomial, and whether the word is an unmarked prototype.

\subsection{Name length}

One of the ways frequency of observation could influence the form of the name of a bird is if it corresponds to frequency of use in language. If this were the case, birds with shorter names are predicted to be the most frequently observed birds, following Zipf's law \cite{zipf1936psycho,zipf1949human}. It is not clear on how to predict the effect of mass on name length, although it seems unlikely that bird size would have an effect on name length.

We analyzed both the frequencies and masses of birds named in Zapotec in relation to the name length of the bird. In a direction opposite to our predictions, birds with longer names had a slight tendency to be more frequent, while having a slight tendency to be of smaller size. However, the coefficients of a linear regression suggest that log frequency and log mass are both weak predictors of log name length. Comparing the linear regression model to alternatives that dropped mass as a predictor significantly impaired model performance ($p = 0.02$) but dropping frequency did not ($p = 0.2$). Akaike information criterion (AIC) scores support the same conclusion.

% See Figure~\ref{fig-freq-namelength}
% \begin{figure}[!ht]
%   \begin{center}
%     \includegraphics[width=0.5\textwidth]{./figures/freq-namelength.png}
%         \caption{Frequency densities of birds named in Zapotec as a function of name length.}
%         \label{fig-freq-namelength}
%   \end{center}
% \end{figure}


\subsection{Compound names}

Compound names exist in Hunn's bird naming data on Zapotec with a dash ('-'). Similar to English compound bird names like ``Hummingbird'' and ``Blackbird'', Zapotec has compound names of the form \textit{mgu\^{\i}n-gui\`{u}u} (``bird'' + ``river'') for the Black Phoebe and \textit{mgu\^{\i}n-rch\v{u}up} (``bird'' + ``whistle'') for the Dusky-capped Flycatcher. Here we further examine names based on whether the Zapotec label is a single word (a monomial) or a compound of multiple words.

% \begin{figure*}
%   \begin{center}
%     \includegraphics[width=0.95\textwidth]{./figures/nameforms-both.png}
%         \caption{Frequency densities (left) and Mass densitities (right) for species named in Zapotec, as function of name form. Monomials are single word bird names and compounds are bird names formed from multiple words.}
%         \label{fig-both-nameforms}
%   \end{center}
% \end{figure*}

% Figure~\ref{fig-both-nameforms} presents the distributions of frequency and size for Zapotec birds with monomial names and compound names.
When analyzing the densities of monomials to compounds, we found both distributions in both cases are similar. Neither environmental variable is a predictor of whether a bird is given a single-word or compound label. The coefficients of a logistic regression suggest that log frequency is a stronger predictor than log mass, and that if the frequency or mass of a species increase, it’s less likely to have a compound name. However, when we compared the full logistic regression model to alternatives that removed either log frequency or log mass as a predictor, analyses suggest that model fit doesn’t become significantly worse when either predictor is dropped ($p=0.46$ for log frequency and $p=0.78$ log mass).  Akaike information criterion (AIC) scores support this conclusion as well. 

\subsection{Prototypes}

Monomial labels have also been linked to prototypes in the categorization literature \cite{berlin1972speculations,berlin2014ethnobiological}. In particular, \citeA{berlin2014ethnobiological} proposes that prototypicality may be due to factors including frequency of occurrence. For simplicity, we examine unmarked prototypes in Hunn's data on Zapotec bird-naming  \cite{hunn2008zapotec}. These are labels for a particular bird that Hunn noted were monomials and were recognized at the same level taxa (had the same folk generic and folk specific names). 

We analyzed six unmarked prototypes from \citeA{hunn2008zapotec}, that is, six particular bird species labeled at the folk-generic level, and their respective folk-specific members. We chose six instances in which it was clear there was a single bird labeled as an unmarked prototype. 

Figure \ref{fig-freq-prototype-all} shows the ranked raw frequency distributions for each of the folk-generic category members per unmarked prototype. The top left bar chart shows the frequencies of vultures in Oaxaca, where we see the prototypical Turkey Vulture (in black) is clearly is more frequent. This trend holds across the other folk-generic categories of unmarked prototypes in Zapotec, with a notable exception for the category of Owls (this category has comparatively low frequency counts). One potentially interesting confound for the Owl category is that Hunn notes the unmarked prototype, the Great Horned Owl, is considered an ill omen by many. 

% 24) dǎm̲ [ON] ^[[SndMhpBub1]]; large owls (Strigiformes in part):
% 24a) dǎm̲[-0] [ON, unmarked prototype] (= dǎm̲-rǒb [`owl´ + `great´], dǎm̲-yòx [`owl´ + `great´]); Great Horned Owl (Bubo virginianus): occurs in woodland and forest near town; the generic name is onomatopoetic; it is cognate with Cordova´s tàma, “paxaros que tenian por agueros”; it is considered by many an ill omen; other species of larger owls which might occur here include Mottled (Strix virgata) and Stygian Owls (Asio stygius), though these may be distinguished as the next;
% 24b) dǎm̲-yêt [`owl´ + `small´]; Mottled (Strix virgata) and/or Stygian Owls (Asio stygius): neither has been recorded in San Juan Gbëë, but both are possible; both are smaller than prototypical dǎm̲ (Bubo virginianus), as the modifier implies.


% 5) pěch [`X´] ^[[SndMhpPch1]]: Cathartidae, New World vultures:
% 5a) pěch-msìdòo [`vulture´ + `eagle´] (= rên [`blood´]); King Vulture (Sarcoramphus papa): now, at least, very rare; might be known from travels to the Isthmus of Tehuántepec; one older man reported the name and that it did occur in San Juan Gbëë; the generic synonym was noted by Reeck (1991);
% 5b) pěch-rúx [`vulture´ + `naked´] (= ngól̲-běts [X + Y]); Black Vulture (Coragyps atratus): less common and widespread than the next; the generic synonym was reported by Reeck;
% 5c) pěch[-0] [`vulture´, unmarked prototype] (= pěch-yèts [`vulture´ + `yellowish´]); Turkey Vulture (Cathartes aura): the prototypical vulture; normally simply named pěch.


\begin{figure*}[ht!]
  \begin{center}
    \includegraphics[width=0.95\textwidth]{./figures/prototypes-barplots-all.png}
        \caption{Frequency bar plots of birds named in Zapotec as unmarked prototypes and the frequencies of other birds with the same folk-generic name. The bar highlighted in black is the unmarked prototype.}
        \label{fig-freq-prototype-all}
  \end{center}
\end{figure*}






\section{Discussion}

% Summary of some of the questions we could explore using the methods detailed above.

Questions of how categories of natural kinds are named have a long history in anthropology and psychology. There has been a particular debate over the influence of the external environment on how people partition the world. Recent large-scale digital datasets allow us to test old questions in a new light. In this paper we use frequency of occurrence and physical size of birds to examine the naming patterns of a particular folk categorization system. We found evidence that frequency predicts whether or not a species is named, and when a species is an unmarked prototype, as well as weak evidence that it predicts lumping. We replicated \citeA{hunn1999size} and found size predicts lumping (and find that it's a stronger predictor than frequency), but found only weak evidence that it predicts anything else.

Reasons why we don’t see a stronger effect of frequency (discuss wrt theories that do predict frequency effects including functional/utilitarian view, Zipfian view, perhaps others)

Distribution of species in morphospace is an important confound in lumping, compound vs monomial and namelength analyses. Possible that similarity is the big thing: e.g if a species is very similar to many other species than we’d expect to have a high lumping score independent of mass and frequency, and might also expect to have a compound name that reflects the relationship of the species to other similar things.

% We also address potential concerns that can arise in using eBird frequency of observation data here. Does frequency of observation in eBird accurately represent the statistic of interest? (SOME CITATIONS to back up this claim). Also: These questions are interesting because we typically take for granted the categories of natural kinds. However, scientific taxonomies are just another human-constructed category system. When considering the set of birds in particular, it has been difficult for biologists to agree on a standardized taxonomy, which has been shown to severely impact decisions on conservation policy \cite{peterson2006taxonomy,garnett2017taxonomy}.

\subsection{Future Directions}

We plan to include more environmental features in our analyses. Recall our observation about the owl prototypes above: this was an interesting insight which could have been found by adding additional environmental feature data from EltonTraits \cite{wilman2014eltontraits}. For example, there are features which would indicate that these birds were all nocturnal, and give rise to the natural question: do all norturnal birds logged in eBird have very low frequency of observation like Owls?

Future direction: include similarity as predictor. Use the data in that recent paper (Macroevolutionary convergence connects morphological form to ecological function in birds) \cite{pigot2020macroevolutionary} to add a “neighborhood density” variable that captures how many other local species a given species is similar to.


Another clear next step forward would be to expand these analyses to more languages. To do this one needs to find trustworthy ethnographies similar to the Zapotec naming data we used here from \citeA{hunn2008zapotec}, and one needs decent coverage in eBird over the geographic region in question. Clear next steps would be to analyze the Tzeltal language from Chiapas, Mexico, and the Tlingit language from the south-east Alaska, both published by Hunn as well \cite{hunn1977tzeltal,hunn2012tlingit}, which have decent coverage within their respective geographics regions in eBird observational data. 

That said, it can be difficult to find languages with both expert ethnographries of the folk biological naming systems which also have good coverage in eBird. This has prohibited us from exploring bird naming data from known experts in regions with low coverage in eBird (e.g., naming data summarized in \cite{holman2002relation}, including the Tobelo language from Indonesia \cite{taylor1990folk} and the Anindilyakwa language from Australia \cite{waddy1988classification}, which do not have coverage in eBird currently).


\section{Conclusion}




% \section{Formalities, Footnotes, and Floats}


% The entire content of a paper (including figures, references, and anything else) can be no longer than six pages in the \textbf{initial submission}. In the \textbf{final submission}, the text of the paper, including an author line, must fit on six pages. Up to one additional page can be used for acknowledgements and references.

% The text of the paper should be formatted in two columns with an
% overall width of 7 inches (17.8 cm) and length of 9.25 inches (23.5
% cm), with 0.25 inches between the columns. Leave two line spaces
% between the last author listed and the text of the paper; the text of
% the paper (starting with the abstract) should begin no less than 2.75 inches below the top of the
% page. The left margin should be 0.75 inches and the top margin should
% be 1 inch.  \textbf{The right and bottom margins will depend on
%   whether you use U.S. letter or A4 paper, so you must be sure to
%   measure the width of the printed text.} Use 10~point Times Roman
% with 12~point vertical spacing, unless otherwise specified.

% The title should be in 14~point bold font, centered. The title should
% be formatted with initial caps (the first letter of content words
% capitalized and the rest lower case). In the initial submission, the
% phrase ``Anonymous CogSci submission'' should appear below the title,
% centered, in 11~point bold font.  In the final submission, each
% author's name should appear on a separate line, 11~point bold, and
% centered, with the author's email address in parentheses. Under each
% author's name list the author's affiliation and postal address in
% ordinary 10~point type.

% Indent the first line of each paragraph by 1/8~inch (except for the
% first paragraph of a new section). Do not add extra vertical space
% between paragraphs.


% \section{First Level Headings}

% First level headings should be in 12~point, initial caps, bold and
% centered. Leave one line space above the heading and 1/4~line space
% below the heading.


% \subsection{Second Level Headings}

% Second level headings should be 11~point, initial caps, bold, and
% flush left. Leave one line space above the heading and 1/4~line
% space below the heading.


% \subsubsection{Third Level Headings}

% Third level headings should be 10~point, initial caps, bold, and flush
% left. Leave one line space above the heading, but no space after the
% heading.


% \section{Formalities, Footnotes, and Floats}

% Use standard APA citation format. Citations within the text should
% include the author's last name and year. If the authors' names are
% included in the sentence, place only the year in parentheses, as in
% \citeA{NewellSimon1972a}, but otherwise place the entire reference in
% parentheses with the authors and year separated by a comma
% \cite{NewellSimon1972a}. List multiple references alphabetically and
% separate them by semicolons
% \cite{ChalnickBillman1988a,NewellSimon1972a}. Use the
% ``et~al.'' construction only after listing all the authors to a
% publication in an earlier reference and for citations with four or
% more authors.


% \subsection{Footnotes}

% Indicate footnotes with a number\footnote{Sample of the first
% footnote.} in the text. Place the footnotes in 9~point font at the
% bottom of the column on which they appear. Precede the footnote block
% with a horizontal rule.\footnote{Sample of the second footnote.}


% \subsection{Tables}

% Number tables consecutively. Place the table number and title (in
% 10~point) above the table with one line space above the caption and
% one line space below it, as in Table~\ref{sample-table}. You may float
% tables to the top or bottom of a column, and you may set wide tables across
% both columns.

% \begin{table}[H]
% \begin{center} 
% \caption{Sample table title.} 
% \label{sample-table} 
% \vskip 0.12in
% \begin{tabular}{ll} 
% \hline
% Error type    &  Example \\
% \hline
% Take smaller        &   63 - 44 = 21 \\
% Always borrow~~~~   &   96 - 42 = 34 \\
% 0 - N = N           &   70 - 47 = 37 \\
% 0 - N = 0           &   70 - 47 = 30 \\
% \hline
% \end{tabular} 
% \end{center} 
% \end{table}


% \subsection{Figures}

% All artwork must be very dark for purposes of reproduction and should
% not be hand drawn. Number figures sequentially, placing the figure
% number and caption, in 10~point, after the figure with one line space
% above the caption and one line space below it, as in
% Figure~\ref{sample-figure}. If necessary, leave extra white space at
% the bottom of the page to avoid splitting the figure and figure
% caption. You may float figures to the top or bottom of a column, and
% you may set wide figures across both columns.

% \begin{figure}[H]
% \begin{center}
% \fbox{CoGNiTiVe ScIeNcE}
% \end{center}
% \caption{This is a figure.} 
% \label{sample-figure}
% \end{figure}


% \section{Acknowledgments}

% In the \textbf{initial submission}, please \textbf{do not include
%   acknowledgements}, to preserve anonymity.  In the \textbf{final submission},
% place acknowledgments (including funding information) in a section \textbf{at
% the end of the paper}.


% \section{References Instructions}

% Follow the APA Publication Manual for citation format, both within the
% text and in the reference list, with the following exceptions: (a) do
% not cite the page numbers of any book, including chapters in edited
% volumes; (b) use the same format for unpublished references as for
% published ones. Alphabetize references by the surnames of the authors,
% with single author entries preceding multiple author entries. Order
% references by the same authors by the year of publication, with the
% earliest first.

% Use a first level section heading, ``{\bf References}'', as shown
% below. Use a hanging indent style, with the first line of the
% reference flush against the left margin and subsequent lines indented
% by 1/8~inch. Below are example references for a conference paper, book
% chapter, journal article, dissertation, book, technical report, and
% edited volume, respectively.

% \nocite{ChalnickBillman1988a}
% \nocite{Feigenbaum1963a}
% \nocite{Hill1983a}
% \nocite{OhlssonLangley1985a}
% % \nocite{Lewis1978a}
% \nocite{Matlock2001}
% \nocite{NewellSimon1972a}
% \nocite{ShragerLangley1990a}

% \newpage
\bibliographystyle{apacite}

\setlength{\bibleftmargin}{.125in}
\setlength{\bibindent}{-\bibleftmargin}

\bibliography{CogSci2020}


\end{document}
